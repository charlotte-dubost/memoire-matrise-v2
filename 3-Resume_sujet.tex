% Résumé du mémoire.
%
\chapter*{RÉSUMÉ}\thispagestyle{headings}
\addcontentsline{toc}{compteur}{RÉSUMÉ}

La roue Cyr est un agrès de cirque composé d’une roue métallique à la géométrie torique dont le diamètre avoisine la hauteur de l’utilisateur et dont le poids varie de 10kg à 20kg. L’utilisateur se place au centre de la roue, s’y agrippe et effectue des figures acrobatiques tout en tournant. Ceci fait du poids et de la rigidité des facteurs déterminants pour la performance de l’utilisateur. Depuis son invention cet agrès a été revisité plusieurs fois par des artistes et des fabricants d’équipement de cirque. En particulier, depuis que l’utilisation de matériaux avancés (comme la fibre de carbone) est devenue populaire pour les équipements sportifs, et qu’elle commence à être adopté par certains fabricants pour les mats et les portiques, la communauté circassienne manifeste de l'intérêt pour les possibilités nouvelles qu’une roue Cyr plus légère et plus flexible pourrait apporter. La géométrie torique d’une roue Cyr rend sa fabrication plus compliquée et le prototypage plus couteux, parmi les projets de roue Cyr en matériaux composites, aucun n’a été mené à bien, pour cause manque de moyens où d’absence de modélisation théorique. En outre, développer une modélisation dynamique de la roue Cyr (dont le mouvement est similaire à celui du disque d’Euler) nécessite de se plonger dans une étude approfondie.
Le but de ce projet est donc de déterminer de quelle manière l’utilisation de matériaux composites pour la fabrication d’une roue Cyr enrichira la discipline, aussi bien en termes de possibilités acrobatiques que de jeu de scène, ainsi que d’identifier des propriétés mécaniques et géométries optimisées correspondant aux mouvements et aux efforts exercés sur une roue Cyr.



Le déroulement du projet est basé sur la méthodologie suivante:
\begin{itemize}
\item Étude théorique : Développement d’un modèle mathématique des mouvements caractéristiques de la roue Cyr et du stockage d’énergie, des sauts et rebonds. Une fois que le modèle est établi, caractérisation des paramètres géométriques et mécaniques déterminants ainsi que de leur influence. Développement d’une analyse adimensionnelle et d’une loi d’échelle.
\item Étude numérique : développement d’un modèle numérique des mouvements de la roue Cyr en utilisant Python.
\item Preuve de concept : Une fois que les propriétés optimales ont été déterminées imprimer en 3D une petite portion de la roue, de même rayon de courbure, avec le matériau correspondant. Une fois que cette portion est testée et validée, imprimer un modèle réduit et le tester adéquatement à l’aide de la loi d’échelle déterminée à l’étude théorique.
\item Prototypage à l’échelle 1
\item Test du prototype : En partenariat avec l’École Nationale de Cirque de Montréal.
\end{itemize}

Le projet contribue à l’avancement des connaissances dans différents domaines et pour des communautés diversifiées : en plus de compléter certaines recherches concernant le disque d’Euler, ou plus globalement les disques et les tores en rotation ou les anneaux élastiques, ce qui fait l’originalité et l’importance du projet réside dans l’apport de nouvelles connaissances à la communauté circassienne : les fabricants d’équipement de cirque auront une réponse concernant l’intérêt de la fabrication de roue Cyr en matériaux composites, ainsi qu’un exemple de procédé de fabrication et, (surtout !) pour les artistes et metteurs en scène la discipline se verra enrichie par de nouvelles possibilités et de nouvelles opportunités de création.

